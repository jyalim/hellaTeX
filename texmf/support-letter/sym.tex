
\def\pfp{{\small\hfill$\square$}}
\def\pf{{\small\hfill$\blacksquare$}}

\renewcommand{\qedsymbol}{{\small\hfill$\join$}} % \S

\def\pRT{\vv{r},t}
\def\condeq{\rho(\pRT) \; c_p(\pRT) \; \pPD{T(\pRT)}{t} = 
  \pLpa{k(\pRT)}{\nabla T(\pRT)} + H(\pRT)}
\def\heatfloweq{\vv{q}(\pRT) = -k(\pRT)\;\nabla T(\pRT)}

% Jacobi Elliptic Functions
\def\cn{\text{cn}}
\def\sn{\text{sn}}
\def\dn{\text{dn}} 

\def\ot{\otimes}
\def\os{\oplus}

\def\nin{\not\in}

\newcommand{\bdt}[1]{\overset{\bm .}{#1}}
\newcommand{\bddt}[1]{\overset{\bm ..}{#1}}
\newcommand{\bdddt}[1]{\overset{\bm ...}{#1}}

\newcommand{\mtrx}[1]{\bm{#1}} %\underline{\underline{#1}}}
\newcommand{\vctr}[1]{\underline{#1}} %\underline{#1}}

\newcommand{\matr}[1]{\bm{#1}} %\underline{\underline{#1}}}
\newcommand{\vect}[1]{\bm{#1}} %\underline{#1}}

\newcommand{\bvect}[1]{%
  \begin{bmatrix*}[r]
    #1 
  \end{bmatrix*}
}

\newcommand{\bvectc}[1]{%
  \begin{bmatrix*}[c]
    #1 
  \end{bmatrix*}
}

\newcommand{\pvect}[1]{%
  \begin{pmatrix*}[r]
    #1 
  \end{pmatrix*}
}

\newcommand{\pvectc}[1]{%
  \begin{pmatrix*}[c]
    #1 
  \end{pmatrix*}
}

\renewcommand{\vec}[1]{\ensuremath{\mathchoice
  {\mbox{\boldmath$\displaystyle\mathbf{#1}$}}
  {\mbox{\boldmath$\textstyle\mathbf{#1}$}}
  {\mbox{\boldmath$\scriptstyle\mathbf{#1}$}} 
  {\mbox{\boldmath$\scriptscriptstyle\mathbf{#1}$}}}
}

\def\checkmark{\tikz\fill[scale=0.4]%
  (0,.35)--(.25,0)--(1,.7)--(.25,.15)--cycle;
} 

\def\slashmark{$\times$}
\def\pd{\partial}
\newcommand{\pPD}[2]{\frac{\partial{#1}}{\partial{#2}}}                   % ORDER 1 PARTIAL
\newcommand{\pSPD}[3]{\frac{\partial^{#3}{#1}}{\partial{#2}^{#3}}}        % ORDER S PARTIAL
\newcommand{\pD}[2]{\frac{d{#1}}{d{#2}}}                                  % ORDER 1 STANDARD
\newcommand{\pMD}[2]{\frac{D{#1}}{D{#2}}}                                 % ORDER 1 Total
\newcommand{\pSD}[3]{\frac{d^{#3}{#1}}{d{#2}^{3}}}                        % ORDER S STANDARD
\newcommand{\pGrad}[1]{\pPD{#1}{x} + \pPD{#1}{y} + \pPD{#1}{z}}           % EVAL GRADIENT
\newcommand{\pLapl}[1]{\pSPD{#1}{x}{2} + \pPD{#1}{y}{2} + \pPD{#1}{z}{2}} % EVAL LAPLACIAN 
\newcommand{\pSDPR}[2]{\nabla \cdot #1 #2}                                % STANDARD DIV
\newcommand{\pEDPR}[2]{#2 \cdot \nabla #1 + #1 \nabla \cdot #2 }          % EVAL DIV
\newcommand{\pLpc}[2]{#1 \nabla^2 #2}               % STANDARD LAPLACIAN
\newcommand{\pEV}[3]{\left. #1 \right|_{#2}^{#3}}   % EVAL FROM #2 to #3
\newcommand{\pIP}[1]{\left\langle #1 \right\rangle} % INNER PRODUCT
%\newcommand{\abs}[1]{\left\vert #1 \right\vert}     % ABS
\newcommand{\paren}[1]{\left( #1 \right)}           % Big Parenthesis
\newcommand{\vectr}[2]{{#1 \choose #2}}             % Quick Vector Fmt
\newcommand{\tr}[1]{\text{tr}(#1)}                  % Trace
\newcommand{\Iv}[1]{\mathcal{I}_{#1}}               % Invariants

\DeclarePairedDelimiter\abs{\lvert}{\rvert}
\DeclarePairedDelimiter\norm{\lvert\lvert}{\rvert\rvert}
\DeclarePairedDelimiter\snorm{\lvert}{\rvert}
\DeclarePairedDelimiter\pnorm{\lparen}{\rparen}
\DeclarePairedDelimiter\bnorm{\lbrack}{\rbrack}
\DeclarePairedDelimiter\cnorm{\lbrace}{\rbrace}
\DeclarePairedDelimiter\anorm{\langle}{\rangle}
\DeclarePairedDelimiter\vnorm{\lvert}{\rvert}

\newcommand{\sinp}[1]{\sin\pnorm*{#1}}
\newcommand{\cosp}[1]{\cos\pnorm*{#1}}
\newcommand{\tanp}[1]{\tan\pnorm*{#1}}
\newcommand{\psin}[1]{\sin\pnorm*{#1}}
\newcommand{\pcos}[1]{\cos\pnorm*{#1}}
\newcommand{\ptan}[1]{\tan\pnorm*{#1}}

\newcommand{\Psinp}[2]{\sin^{#1}\pnorm*{#2}}
\newcommand{\Pcosp}[2]{\cos^{#1}\pnorm*{#2}}
\newcommand{\Ptanp}[2]{\tan^{#1}\pnorm*{#2}}

\def\R{\mathbb{R}}
\def\N{\mathbb{N}}
\def\Z{\mathbb{Z}}
\def\C{\mathbb{C}}
\def\Q{\mathbb{Q}}
\def\B{\mathbb{B}}

\def\P{\mathbb{P}}

\def\o{\mathcal{o}}
\def\O{\mathcal{O}}

\def\del{\nabla}

\def\pRT{\vv{r},t}
\def\condeq{\rho(\pRT) \; c_p(\pRT) \; \pPD{T(\pRT)}{t} = 
  \pLpa{k(\pRT)}{\nabla T(\pRT)} + H(\pRT)}
\def\heatfloweq{\vv{q}(\pRT) = -k(\pRT)\;\nabla T(\pRT)}

% Jacobi Elliptic Functions
\def\cn{\text{cn}}
\def\sn{\text{sn}}
\def\dn{\text{dn}} 

\def\ot{\otimes}

\newcommand*{\matminus}{%
  \leavevmode
  \hphantom{0}%
  \llap{%
    \settowidth{\dimen0 }{$0$}%
    \resizebox{1.1\dimen0 }{\height}{$-$}%
  }%
} % From http://tex.stackexchange.com/questions/75545/
  % ... negative-sign-and-matrix-alignment
  % User: Heiko Oberdiek
\def\mm{\matminus}

\DeclareMathOperator*\rank{rank}
\DeclareMathOperator*\nullity{nullity}
\DeclareMathOperator*\range{range}
\DeclareMathOperator*\Span{span}         % \span conflicts
\DeclareMathOperator*\codomain{codomain}
\DeclareMathOperator*\domain{domain}
\DeclareMathOperator*\dom{dom}
\DeclareMathOperator*\image{image}
\DeclareMathOperator*\coimage{coimage}
\DeclareMathOperator*\argmin{argmin}
\DeclareMathOperator*\argmax{argmax}
\DeclareMathOperator*\sign{sign}
\DeclareMathOperator*\Li{Li}
\DeclareMathOperator*\Ci{Ci}
\DeclareMathOperator*\Si{Si}

\DeclarePairedDelimiter\ceil{\lceil}{\rceil}
\DeclarePairedDelimiter\floor{\lfloor}{\rfloor}

% ======================================================================
% From 
% http://tex.stackexchange.com/questions/112161/bitcoin-symbol-in-latex
% ----------------------------------------------------------------------

\def\bitcoinA{%
  \leavevmode
  \vtop{ %
    \offinterlineskip %\bfseries
    \setbox0=\hbox{\bf B} %
    \setbox2=\hbox to\wd0{%
      \hfil\hskip-.10em
      \vrule height .3ex width .15ex\hskip .08em
      \vrule height .3ex width .15ex\hfil
    }
    \vbox{\copy2\box0}\box2
  }
}

\def\bitcoinB{\leavevmode{\setbox0=\hbox{\textsf{B}}%
    \dimen0\ht0 \advance\dimen0 0.2ex
    \ooalign{\hfil \box0\hfil\cr
      \hfil\vrule height \dimen0 depth.2ex\hfil\cr
    }%
  }%
}

\def\bitcoinC{\leavevmode\rlap{\hskip.5pt-}B} 

% ======================================================================
% From 
% ctan.org/info/svg-inkscape/InkscapePDFLaTeX.pdf 
% https://bitcointalk.org/index.php?topic=96438.0
% with svg provided from 
% https://en.bitcoin.it/wiki/Bitcoin_symbol
% ----------------------------------------------------------------------

\newcommand*\btc{%
  \def\svgwidth{6pt}
  \input{support/var/btc.pdf_tex}
}

\newcommand*\BTC{\bitcoinA}


\newcommand*\ETH{Ξ}

\input{support/var/varsymb.tex}
